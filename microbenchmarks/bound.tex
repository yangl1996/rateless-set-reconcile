\documentclass{article}
\begin{document}

\noindent
\textbf{Algorithm.} There are two parameters: $t$ (detection threshold), and
$c$ (control overhead). Each node is free to choose these parameters for
incoming blocks, and these parameters only affect the overhead (discussed
below) of this node. Peers sending to this node are expected to honor these
parameters (but it is easy for the receiving node to verify and enforce
the parameters). We now discuss the algorithm for sending transactions to peers.

\smallskip
For each peer, a node maintains a sending buffer. The node inserts every new
transaction (either decoded or locally generated) into the buffer, and makes
sure it never inserts the same transaction into the buffer more than once.
When the node finishes sending the current block (discussed next) and there
are at least $t/c$ transactions in the sending buffer, it takes the entire
sending buffer as the next block and starts sending it.

\smallskip
To send the block, the node generates coded symbols using LT codes and sends
them to the peer. The peer attempts to decode these coded symbols using the
peeling decoder. It tracks whether each received coded symbol has been decoded.
If it has received at least $t$ coded symbols for the current block, and all
coded symbols received for the block so far have been decoded, then it decides
that the entire current block has been successfully decoded and notifies the
sending peer. The sender moves onto the next block upon receiving the
acknowledgement, as discussed above.


\smallskip\noindent\textbf{Analysis.}
Let $D$ be the number of peers. Let $N_i$ for $1 \le i \le D$ be the number of
blocks that the $i$-th peer sends to the receiver for the lifetime of the
system. For convenience, let $N = \sum_{i=1}^D N_i$, i.e., $N$ is the number of
blocks that the receiver receives.  Let $x_j$ for $1 \le j \le N$ be the number
of \emph{fresh} transactions in the $j$-th block that the receiver decodes, and
let $x=\sum_{j=1}^N x_j$ for convenience.  Finally, let $B$ be the block size,
and $f(t)$ be the number of codewords required to send a block with $t$ fresh
transactions. Note that the receiver needs to receive $f(x_j)$ codewords in
order to decode the $j$-th block. The system overhead is defined as
$\sum_{j=1}^N f(x_j)/x$, which is the average number of codewords needed to
decode a fresh transaction.

\smallskip
\noindent
\textbf{Lemma 1.} $x \ge NB/D.$

\smallskip
\noindent \textbf{Proof.} The algorithm guarantees that blocks from the same
peer are disjoint, so peer $i$ sends $N_i B$ fresh transactions, i.e., $x \ge
N_i B$ for $1 \le i \le D$. Because by definition $N=\sum_{i=1}^D N_i$, there
exists at least one $i^*$, such that $1 \le i^* \le D$ and $N_{i^*} \ge N/D$.
Plug $i^*$ into $x \ge N_i B$ and we immediate get $x \ge NB/D$.

\smallskip
\noindent
\textbf{Theorem 1.} If $f(t)$ is concave and $f(t)/t$ is monotonically decreasing, then $$\frac{\sum_{j=1}^N f(x_j)}{x} \le \frac{D}{B}f\left(\frac{B}{D}\right).$$ 

\smallskip \noindent \textbf{Proof.} By Jensen's inequality and the assumption
that $f(t)$ is concave, $$\sum_{j=1}^N \frac{f(x_j)}{N} \le
f\left(\frac{\sum_{j=1}^N x_j}{N}\right) = f\left(\frac{x}{N}\right).$$ Then
$$\frac{\sum_{j=1}^N f(x_j)}{x} \le \frac{N}{x}f\left(\frac{x}{N}\right).$$ By
Lemma 1 and the assumption that $f(t)/t$ monotonically decreases,
$$\frac{N}{x}f\left(\frac{x}{N}\right) \le
\frac{D}{B}f\left(\frac{B}{D}\right).$$ 

\smallskip \noindent \textbf{Discussion.} The theorem says that the average
number of codewords that the receiver needs to download in order to decode a
fresh transaction is upper-bounded by $\frac{D}{B}f\left(\frac{B}{D}\right)$.
This upper bound is achieved when every block contains $D/B$ fresh transactions.
Here are some sample calculations for $B=500$.

\begin{center}
\begin{tabular}{ r|r }
    \hline
    $D$ & max. overhead\\\hline
    2 & $1.65$ \\
    3 & $2.20$ \\
    4 & $2.67$ \\
    5 & $3.17$ \\
    6 & $3.70$ \\
    7 & $4.11$ \\
    8 & $4.39$ \\
    9 & $4.80$ \\
    10 & $5.02$ \\
 \hline
\end{tabular}
\end{center}

\end{document}
